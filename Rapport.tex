\documentclass[12pt]{article}
\usepackage[utf8]{inputenc}
\usepackage[T1]{fontenc}
\usepackage[french]{babel}
\usepackage{amsmath,amsfonts,amssymb}
\usepackage{fullpage}
\usepackage{graphicx}
\usepackage{hyperref}
\title{Diagrammes de décision binaires}
\author{Jimmy Rogala \& Nathan Thomasset}

\begin{document}

\maketitle

\section*{Question 1}

Soit $ v : \text{Var}(\varphi) \to \{0, 1\}$. On cherche à montrer $v(\varphi) = v(\varphi\uparrow^{x})$ (on utilise de la même manière $v$ et son extension inductive à l'ensemble des formules pour plus de lisibilité). \\
On a (par une induction triviale) $\forall x \in \text{Var}(\varphi), v(\varphi)=v(\varphi[v(x)/x])$. \\
Supposons $v(x)=1$. \\
Alors $\varphi\uparrow^{x} \equiv \varphi[1/x]$ d'où $v(\varphi\uparrow^{x})=v(\varphi[1/x])=v[\varphi(v(x)/x]=v(\varphi)$. \\
De même si $v(x)=0$ on obtient $v(\varphi\uparrow^{x})=v(\varphi[0/x])=v(\varphi)$. \\
Toutes les valuations coïncident bien en $\varphi$ et $\varphi\uparrow^{x}$, d'où $\varphi \equiv \varphi\uparrow^{x}$.


\end{document}
